\newpage
\section{Conclusioni}

Dagli esperimenti condotti abbiamo avuto come già detto la conferma di certe implicazioni teoriche. Se da una parte il metodo della divisione risulta più semplice da implementare, dall'altra appare vulnerabile a certi dati posti in ingresso. Alcuni di questi possono rendere alcune celle della tabella intasate in poco tempo e appesantire drasticamente l'esecuzione, come visto nell'esempio \ref{caso_peggiore_div}. Questo porta a fare riflessioni soprattutto nell'ambito della sicurezza informatica: un eventuale attacco dall'esterno potrebbe "dirottare" le chiavi degli elementi in ingresso verso i valori critici per il caso specifico, causando le spiacevoli conseguenze suddette. 
\\ D'altro canto il metodo della moltiplicazione non è critico rispetto al valore della dimensione della tabella, ma richiede tuttavia una scelta appropriata del valore di $A$. Inoltre l'operazione di moltiplicazione risulta più impegnativa per il calcolatore quando la dimensione della tabella non è una potenza di due ed è nettamente più conveniente l'operazione di divisione quando la tabella ha un numero primo come dimensione (viceversa se la dimensione è una potenza di due l'operazione di moltiplicazione equivale a un semplice AND bit a bit, quindi semplice e veloce in tal caso) \cite{Stack}.

Detto ciò, possiamo dire che su larga scala i due metodi si comportano sostanzialmente allo stesso modo, come abbiamo visto nel caso di Figura \ref{fig:random_casi}, pertanto a seconda delle esigenze si può scegliere l'uno o l'altro. Necessario però sottolineare che oggigiorno esistono metodi e funzioni più avanzate per ridurre le collisioni come l'hash perfetto.