\section{Risultati attesi} \label{ris_attesi}
Dalle implicazioni teoriche che abbiamo esposto in particolare nella sezione \ref{metodo_div} ci aspettiamo che il metodo della divisione produca dei risultati negativi (ovvero numerose collisioni) quando $m$ è una potenza intera di due oppure per come è definita $h(k)$ quando abbiamo tanti valori delle chiavi $k$ che sono multipli di $m$.
\\Risulta invece difficile prevedere a priori l'esatta evoluzione della tabella quando si utilizza il metodo della moltiplicazione. Sappiamo però che qualsiasi valore di $A \in (0;1)$ è sufficiente buono, anche se quello di Knuth sembra il più appropriato \cite{textbook}\cite{fibo_hash}.

Per quanto riguarda in generale il \textit{load factor}, ci aspettiamo che esso influisca sulle prestazioni quando diventa un valore superiore al $75 \%$ \cite{wikipedia}.