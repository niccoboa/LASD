\section{Descrizione del lavoro svolto} \label{descr_lavoro}

Per completare l'esercizio sono stati effettuati diversi test volti a evidenziare il comportamento delle tabelle a seconda di specifici dati in ingresso.

L'approccio seguito in primo luogo è stato quello del \textit{worst case analysis}, volto a studiare cioè l'evoluzione delle tabelle hash quando in ingresso vengono inseriti i dati considerati appunto \textit{peggiori} per i relativi metodi.
\\ Successivamente lo studio si è concentrato su una analisi più generale e casuale, con l'obiettivo di riassumere al meglio i vantaggi e gli svantaggi di ogni metodo quando in input si hanno dati non predicibili.
\\Osserveremo allo stesso tempo il progressivo riempimento della tabella man mano che cresce il \textit{load factor} $\alpha$.

\subsection{Sviluppo e implementazione dei metodi}
I test sono stati effettuati in \verb|Python v3.10| utilizzando \verb|Matplotlib| come libreria per la creazione dei grafici.
\\Al fine di confrontare al meglio i due metodi è stata creata una tabella apposita per ognuno: \verb|T1| è riempita da oggetti della classe \verb|User| attraverso il metodo della divisione; \verb|T2| ospita i medesimi oggetti della classe \verb|User| ma questi sono inseriti sfruttando il metodo della moltiplicazione\footnote{In alcuni esempi (come il \ref{worst_case_mol}), sono state utilizzate più di due tabelle per completare il confronto}.
\\ Ogni oggetto è composto dall'abbinamento degli attributi \verb|key| e \verb|value|, dove quest'ultimo è un identificatore \textit{human-readable}, cioè una stringa caratteristica dell'oggetto stesso (facoltativo ai fini degli esperimenti ma utile per la comprensione e per possibili implementazioni più complesse).
\\ Le principali funzioni del programma sono:
\begin{itemize}
    \item \verb|func_div(key)| e \verb|func_mul(key)|: Funzioni hash
    \item \verb|insert(User)|: Inserimento simultaneo di un utente in \verb|T1| e \verb|T2|
    \item \verb|delete(User)|: Rimozione di un utente da \verb|T1| e \verb|T2|
    \item \verb|searchT1(key)|: Ricerca chiave in \verb|T1| (analoga per \verb|T2|)
    \item \verb|print_all()|: Visualizzazione su console delle tabelle, del \textit{load factor} e del numero di collisioni rilevate su ciascuna tabella
    \item \verb|plt.show()|: Visualizzazione dei grafici sulla GUI di \verb|Matplotlib|
\end{itemize}


\subsection{Ipotesi}
Le ipotesi che assumeremo vere sono le seguenti:
\begin{enumerate}
    \item[i.] Non possono esserci due \verb|User| con la stessa chiave;
    \item[ii.] Le collisioni sono gestite con il \textit{concatenamento};
    \item[iii.] Le tabelle hanno tutte la stessa dimensione $m$.
\end{enumerate}



\begin{comment}
    \lstset{
      basicstyle=\footnotesize,
      xleftmargin=.3\textwidth, xrightmargin=.2\textwidth
    }
    \begin{lstlisting}
        insert(T,100,"Greco")
    \end{lstlisting}
    
    Abbiamo poi ottenuto il seguente grafico che mostra...
    
    \begin{figure}[htb]
    \begin{center}
    \includegraphics[scale=0.5]{src/img/graph1.pdf}
    \caption{This is a figure.}
    \end{center}
    \end{figure}
\end{comment}